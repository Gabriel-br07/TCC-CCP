%%%%%%%%%%%%%%%%%%%%%%%%%%%%%%%%%%%%%%%%%%%%%%%%%%%%%%%%%%%%%%%%%%%%%%%%%%%%%%%
% Seção: Módulo de Indexação Vetorial (DBVECTOR)
%%%%%%%%%%%%%%%%%%%%%%%%%%%%%%%%%%%%%%%%%%%%%%%%%%%%%%%%%%%%%%%%%%%%%%%%%%%%%%%

\section{Módulo de Indexação Vetorial (DBVECTOR)}
\label{sec:dbvector}

\subsection{Objetivo e Escopo}
A recuperação eficiente de informações em grandes acervos documentais constitui desafio fundamental em sistemas de recuperação de informação, especialmente quando aplicados ao domínio jurídico, caracterizado por vocabulário especializado e alta densidade informacional. Nesse contexto, o módulo DBVECTOR foi concebido e desenvolvido para prover uma camada especializada de indexação semântica, responsável pela construção, armazenamento e consulta de representações vetoriais (\textit{embeddings}) de documentos textuais, com ênfase em documentos jurídicos relacionados ao SEEU, legislação penal e jurisprudência.

O propósito central desse módulo é permitir a recuperação eficiente dos documentos mais relevantes para uma consulta formulada em linguagem natural, servindo de alicerce para o \textit{pipeline} de Recuperação Aumentada por Geração (RAG). Para tanto, o DBVECTOR articula os seguintes objetivos específicos: (i) receber documentos previamente estruturados em formato textual, acompanhados de metadados enriquecidos; (ii) realizar pré-processamento textual e geração de \textit{embeddings} por meio de modelos de linguagem especializados; (iii) construir e manter índices vetoriais em \textit{backends} especializados, notadamente FAISS e, de forma opcional, OpenSearch; (iv) expor operações de consulta que aceitem \textit{queries} em linguagem natural e retornem documentos ordenados por similaridade semântica; e (v) disponibilizar mecanismos de avaliação da qualidade de recuperação e do desempenho do sistema, permitindo aprimoramentos contínuos.

\subsection{Arquitetura Lógica e Componentes Principais}
A arquitetura lógica do DBVECTOR adota organização em camadas, segregando responsabilidades relacionadas à geração de vetores, ao armazenamento, à consulta e à avaliação. Essa separação de responsabilidades, fundamentada em princípios de engenharia de software, favorece a modularidade, extensibilidade e manutenibilidade do sistema. De forma sintética, destacam-se os seguintes componentes principais:

\begin{itemize}
  \item \textbf{Camada de tratamento de dados}: responsável por limpar, normalizar e, quando necessário, segmentar o texto dos documentos, bem como padronizar campos de metadados antes da vetorização. Essa etapa é crucial para garantir a qualidade dos \textit{embeddings} subsequentes;
  \item \textbf{Camada de \textit{embeddings}}: encapsula a integração com modelos de linguagem --- sejam eles executados localmente ou acessados remotamente via API --- para geração de \textit{embeddings} a partir de trechos de texto, produzindo vetores numéricos em ponto flutuante que capturam relações semânticas;
  \item \textbf{Camada de armazenamento vetorial}: implementa o armazenamento persistente e a busca aproximada por similaridade, com suporte a \textit{backends} especializados como FAISS e, alternativamente, OpenSearch com capacidades vetoriais. A escolha do \textit{backend} influencia diretamente o desempenho e a escalabilidade do sistema;
  \item \textbf{Camada de API e serviços}: expõe interface de consulta (API) que recebe consultas textuais, aciona a geração de \textit{embeddings} da \textit{query}, executa a busca vetorial e retorna os documentos mais similares juntamente com seus metadados enriquecidos, em formato estruturado para consumo pelos demais componentes da \textit{pipeline};
  \item \textbf{Camada de avaliação}: agrega \textit{scripts} e procedimentos voltados à avaliação de qualidade --- por meio de métricas de recuperação como precisão, \textit{recall} e F\textsubscript{1} --- e de desempenho das operações de indexação e consulta, permitindo aprimoramentos contínuos.
\end{itemize}

Essa organização arquitetural permite que alterações em um componente específico --- tais como substituição do modelo de \textit{embeddings} ou troca do \textit{backend} de armazenamento --- sejam realizadas com impacto reduzido nos demais módulos. Por conseguinte, favorece-se a extensibilidade do sistema e facilita-se sua manutenção evolutiva, características essenciais para projetos de pesquisa e desenvolvimento em ambientes dinâmicos.

\subsection{Fluxo de Processamento e Consulta}
O fluxo lógico de dados no DBVECTOR pode ser descrito em duas fases principais: \textit{indexação} e \textit{consulta}.

Na fase de \textbf{indexação}, o sistema recebe como entrada documentos em formato estruturado, normalmente em arquivos JSON Lines (\texttt{.jsonl}), contendo texto e metadados. A partir dessa entrada, o fluxo segue as etapas:

\begin{enumerate}
  \item \textbf{Tratamento}: o texto é submetido a procedimentos de limpeza e normalização, que podem incluir remoção de caracteres indesejados, padronização de espaçamento, deduplicação de trechos e eventual segmentação em unidades menores;
  \item \textbf{Geração de \textit{embeddings}}: cada documento, ou cada segmento textual, é convertido em um vetor de características numéricas por meio de um modelo de linguagem pré-treinado;
  \item \textbf{Construção do índice}: os vetores gerados são inseridos em uma estrutura de índice vetorial (por exemplo, FAISS), a qual é otimizada para operações de busca por vizinhos mais próximos em alta dimensão;
  \item \textbf{Mapeamento entre vetores e metadados}: paralelamente, é mantido um arquivo ou estrutura de metadados que associa cada vetor (identificado por um \textit{id} interno) ao documento original, preservando campos como título, conteúdo, fonte e outros atributos relevantes.
\end{enumerate}

Na fase de \textbf{consulta}, o fluxo é análogo, mas orientado à recuperação:

\begin{enumerate}
  \item o usuário ou sistema cliente envia uma \textit{query} em texto para a API do DBVECTOR;
  \item a \textit{query} é pré-processada e convertida em um vetor de \textit{embedding} pelo mesmo modelo utilizado na indexação;
  \item o vetor da \textit{query} é consultado no índice vetorial, que retorna os identificadores dos vetores mais próximos (maior similaridade);
  \item os identificadores retornados são utilizados para recuperar os documentos e metadados correspondentes;
  \item o conjunto de resultados é ordenado e, se necessário, submetido a um pós-processamento (por exemplo, reordenação ou filtragem) antes de ser devolvido ao cliente.
\end{enumerate}

Esse fluxo garante consistência entre a forma como documentos e consultas são representados, o que é essencial para a qualidade do sistema de recuperação semântica.

\subsection{Formato dos Dados e Esquema Documental}
Os documentos tratados pelo DBVECTOR seguem, em geral, um esquema simples e flexível em formato JSON Lines. Cada linha representa um documento e contém campos típicos como:

\begin{itemize}
  \item \texttt{id}: identificador único do documento;
  \item \texttt{title}: título ou descrição resumida;
  \item \texttt{content}: texto principal do documento (por exemplo, inteiro teor de uma decisão ou conteúdo de um manual);
  \item \texttt{source}: informação sobre a origem (tribunal, sistema, \textit{cluster} temático);
  \item \texttt{metadata}: estrutura adicional com campos específicos (artigo do Código Penal, tipo de documento, data, entre outros).
\end{itemize}

Os \textit{embeddings} são representados internamente como vetores de números em ponto flutuante (tipicamente \texttt{float32}) organizados em matrizes, nas quais cada linha corresponde a um documento ou segmento textual. Embora esses vetores sejam armazenados nos índices vetoriais, o mapeamento entre identificadores internos e documentos originais é mantido em estruturas paralelas, garantindo rastreabilidade e transparência.

\subsection{Estratégias de Indexação e Armazenamento}
O DBVECTOR suporta diferentes estratégias de indexação e armazenamento, com destaque para:

\begin{itemize}
  \item \textbf{FAISS}: utilizado como backend principal para armazenamento local de índices vetoriais de alta performance. Essa solução é adequada para cenários em que o volume de dados é compatível com o ambiente local e se busca baixa latência em consultas;
  \item \textbf{OpenSearch}: empregado como opção alternativa para cenários em que se requerem capacidades adicionais, como busca híbrida (combinação de texto e vetores), replicação, distribuição e integração via APIs REST em ambientes mais próximos de produção.
\end{itemize}

Em ambos os casos, o sistema mantém a associação entre cada vetor armazenado e seus metadados. Isso permite que, após a obtenção de um conjunto de identificadores pelo índice vetorial, a camada de aplicação recupere o documento completo e informações complementares, que são posteriormente utilizadas pelo \textit{pipeline} RAG para compor respostas fundamentadas.

Questões de compatibilidade de versões (especialmente no caso do FAISS) e de capacidade de armazenamento são consideradas na operação do DBVECTOR. Mudanças significativas de versão ou de configuração do backend podem exigir a reconstrução dos índices a partir dos arquivos de documentos originais.

\subsection{API de Consulta e Integração com a Aplicação}
Para integração com os demais componentes do sistema, em especial com a interface e com a camada de orquestração do RAG, o DBVECTOR disponibiliza uma API de consulta. Essa API, em linhas gerais, oferece:

\begin{itemize}
  \item um ponto de entrada para recebimento de \textit{queries} textuais;
  \item rotinas internas para geração de \textit{embeddings} da consulta;
  \item chamadas ao backend vetorial selecionado (FAISS ou OpenSearch) para recuperação dos vetores mais similares;
  \item mapeamento dos resultados para documentos e metadados legíveis pela aplicação cliente;
  \item retorno estruturado, adequado ao consumo por modelos de linguagem ou por outras camadas de apresentação.
\end{itemize}

Dessa forma, o DBVECTOR atua como um serviço especializado de busca semântica, abstrato em relação ao backend de armazenamento, mas consistente em relação ao formato de entrada e saída, o que favorece a integração com a interface descrita anteriormente e com os módulos de coleta de dados.

\subsection{Avaliação, Extensões e Considerações Finais}
O módulo inclui ainda mecanismos voltados à avaliação da qualidade da recuperação, por meio de conjuntos de dados de teste e \textit{scripts} que calculam métricas pertinentes (como \textit{recall} em diferentes profundidades). Esses recursos permitem comparar configurações de índice, modelos de \textit{embeddings} e estratégias de pós-processamento, contribuindo para a escolha das melhores combinações para o domínio jurídico considerado.

Em termos de extensibilidade, a arquitetura do DBVECTOR permite: (i) a adição de novos backends de armazenamento vetorial, por meio de implementações específicas que seguem um contrato comum de armazenamento e busca; (ii) a substituição ou atualização do modelo de \textit{embeddings}, desde que mantida a interface de geração de vetores; e (iii) a criação de novos \textit{pipelines} de ingestão de dados, capazes de produzir documentos no formato JSONL esperado pelo sistema.

Por fim, cabe destacar que a operação do DBVECTOR está sujeita a limitações de ambiente (por exemplo, disponibilidade de GPU ou CPU, restrições de memória e armazenamento) e a cuidados relacionados à privacidade dos dados indexados. Documentos que contenham informações sensíveis devem ser previamente anonimizados ou tratados conforme as políticas de segurança adotadas. Ainda assim, o módulo fornece uma base robusta e flexível para indexação semântica, desempenhando papel central na qualidade das respostas geradas pelo \textit{pipeline} de Recuperação Aumentada por Geração proposto neste trabalho.
