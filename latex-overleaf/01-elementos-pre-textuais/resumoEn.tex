%
% Documento: Resumo (Inglês)
%

\begin{ABSTRACT}
\thispagestyle{empty}
\OnehalfSpacing

\noindent This Final Undergraduate Thesis presents the development of a Retrieval Augmented Generation RAG pipeline applied to the Unified Electronic Penal Execution System SEEU of the National Council of Justice Its objective is to enhance data support retrieval within SEEU by facilitating use by different user profiles on the platform integrating vector databases with a Large Language Model LLM to provide naturallanguage responses grounded in original documents The methodology includes automated collection of official PDF documents preprocessing with cleaning and text segmentation embeddingbased vectorization and indexing in a semantic search engine orchestration via the LangChain library between the FAISS or OpenSearch retrieval engine and the LLM and deployment of the service via chatbot and REST API Evaluation will employ precision recall and F1 score metrics as well as usability testing The aim is to achieve an eighty percent reduction in search time and limit the hallucination rate to five percent Expected outcomes include improved information retrieval efficiency enhanced response quality and alignment with the Justice 40 initiative and UN SDG 16 strengthening transparency and access to justice It is concluded that the proposed architecture enables the digital transformation of penal execution and provides a foundation for future expansions within the Judiciary

\SingleSpacing
\noindent \textbf{Keywords}: penal execution; vector databases; digital transformation.

\end{ABSTRACT}
