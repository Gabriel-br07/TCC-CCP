%
% Documento: Resumo (Português)
%

\begin{RESUMO}
\thispagestyle{empty}
\OnehalfSpacing

\noindent Este Trabalho de Conclusão de Curso apresenta o desenvolvimento de uma pipeline de Geração de Respostas Aumentada por Recuperação de Informação — Retrieval Augmented Generation, RAG — aplicada ao Sistema Eletrônico de Execução Unificado, SEEU, do Conselho Nacional de Justiça. O objetivo é melhorar a obtenção de dados de suporte do SEEU. Facilitando o uso pelos diferentes perfis de usuário da plataforma, integrando bancos de dados vetoriais a um Large Language Model, LLM, de modo a fornecer respostas em linguagem natural fundamentadas em documentos originais. A metodologia abrange a coleta automatizada de documentos oficiais em PDF; o pré-processamento com limpeza e segmentação textual; a vetorização por embeddings e indexação em mecanismo de busca semântica; a orquestração, por meio da biblioteca LangChain, entre o motor de recuperação FAISS ou OpenSearch e o LLM; e a disponibilização do serviço por chatbot e API REST. A avaliação empregará as métricas precisão, recall e F1 score, bem como ensaios de usabilidade. Pretende-se reduzir em oitenta por cento o tempo de busca e limitar a taxa de alucinação a cinco por cento. Os resultados esperados incluem ganho de eficiência na recuperação de informações, melhoria da qualidade das respostas e alinhamento às iniciativas Justiça quatro ponto zero e ODS dezesseis, fortalecendo a transparência e o acesso à Justiça. Conclui-se que a arquitetura proposta viabiliza a transformação digital da execução penal e constitui base para expansões futuras no âmbito do Poder Judiciário.

\SingleSpacing
\noindent \textbf{Palavras-chaves}: execução penal; bancos de dados vetoriais; transformação digital.

\end{RESUMO}

